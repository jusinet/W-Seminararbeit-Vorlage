\subsection{Motivation}
    Die meisten von Ihnen werden ihre Seminararbeit mit Word oder ähnlichen Werkzeugen verfassen. Das ist auch verständlich.
    Word bietet ein eindeutiges WYSIWYG\footnote{What you see is what you get}-Prinzip. Somit können Sie sich sicher sein, dass ihr Dokument genau so aus dem Drucker kommen wird,
    wie sie es auf dem Bildschirm vor sich sehen.
    Dies birgt jedoch einige Fallstricke. Oftmals treten Heulkrämpfe bei der Formatierung auf, und jedes Mal muss nach 'Wie lasse ich die Seitennummerierung ab Seite 3 beginnen?' gegooglet werden.
    
    Abhilfe kann hier \LaTeX bieten. Im Verlauf dieses Textes wird genauer beleuchtet, weshalb \LaTeX möglicherweise die bessere Wahl ist.

\subsection{TeX}
\subsubsection{What you mean is what you get}
    Das Dokument wird in einer formalen Sprache beschrieben. Dadurch kann sichergestellt werden, dass Sie genau das Resultat erwirken, dass Sie in dieser Sprache beschrieben haben.
    Sie schreiben also nicht nur Ihren Text, sondern annotieren ihn mit Informationen zu Fußnoten, Quellen, Bildern, Formeln, usw.
    
\subsubsection{Modularität}
    Sollte Ihnen während der Bearbeitung auffallen, dass Sie doch den Zitierstil wechseln möchten, ist dies mit nur einer geänderten Zeile möglich.
    In Word müssten Sie dafür sämliche Verweise einzeln abändern. Dies funktioniert in \LaTeX komplett automatisch und modular.

\subsection{Anpassungsmöglichkeiten}
Es gibt in dieser Seminararbeitsvorlage mehrere Anpassungsmöglichkeiten. In der Parameterdatei können Sie den Zitatstil, Inhaltsverzeichnisstil, die Sprache\footnote{Dementsprechend werden auch Bezeichnungen für 'Inhaltsverzeichnis' etc. angepasst}
und weitere Verzeichnisse anpassen. Diese Änderungen wirken sich sofort auf das ganze Dokument aus. Sie müssen nichts weiter tun, als Ihren bevorzugten Stil wählen. Den Rest macht \LaTeX.
