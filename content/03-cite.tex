\subsection{Wie zitiere ich?}
    Es gibt mehrere Arten von Zitaten. Die erste Unterscheidung teilt Zitate in Direkte und Indirekte auf.
    Direkte Zitate werden nach ihrer Länge unterschieden. Es gibt das Kürzezitat. Hier wird der Textfluss nicht unterbrochen und das Zitat schmiegt sich in den Fließtext ein.\\
    \emph{Beispiel.} Mein Nachbar ist "[\dots] weder nett zu mir, noch habe ich ihn besonders gerne." \autocite[][S. 69]{ruprecht}\\
    Etwas anders sehen Kurzzitate und Langzitate aus. Sie werden eingerückt und so als Zitat deutlich.\\
    \emph{Beispiel.}\\
    \begin{displayquote}
        Lorem ipsum dolor sit amet, consetetur sadipscing elitr, sed diam nonumy eirmod tempor invidunt ut labore et dolore magna aliquyam erat, sed diam voluptua.
        At vero eos et accusam et justo duo dolores et ea rebum.
        Stet clita kasd gubergren, no sea takimata sanctus est Lorem ipsum dolor sit amet.
        Lorem ipsum dolor sit amet, consetetur sadipscing elitr, sed diam nonumy eirmod tempor invidunt ut labore et dolore magna aliquyam erat, sed diam voluptua.ä
        At vero eos et accusam et justo duo dolores et ea rebum. Stet clita kasd gubergren, no sea takimata sanctus est Lorem ipsum dolor sit amet. \autocite[S. 11f.]{simpleclub}
    \end{displayquote}
    Daneben exisitieren noch indirekte Zitate. So kann beispielsweise auf eine Quelle verwiesen werden.\\
    \emph{Beispiel.} Im Video von SimpleClub \autocite{simpleclub} wird der schiefe Wurf erklärt.\\
    Alternativ kann ein vergleichendes Zitat erstellt werden.\\
    \emph{Beispiel.} Auch in einem Buch wird das gleiche Phänomen beschrieben. \autocite[vgl.][]{ruprecht}\\
    Zudem kann ein direktes Zitat in ein indirektes Zitat umgewandelt werden. Hier wurde das erste Zitat angepasst.\\
    \emph{Beispiel.} Mein Nachbar sei weder nett zu mir, noch habe ich ihn besonders gerne -- so spricht der Autor über seinen Nachbarn \autocite[vgl.][S. 69]{ruprecht}\\
    \vspace{1cm} 
    Es funktioniert in \LaTeX{} so:
    \begin{itemize}
        \item Schreiben Sie Ihre Quelle in die \texttt{quellen.bib}-Datei
        \item Zitieren Sie die Quelle im Text mit \begin{verbatim}
            \autocite[vgl.][S. xff.]{quelle}
        \end{verbatim}
        Lassen Sie dabei das 'vgl.' samt Klammern weg, wenn Ihr Zitat direkt sein soll.
        Die Seitenangabe können Sie entsprechend anpassen bzw. bei Internetquellen oder Referenzen weglassen. Das Vorgehen dabei ist analog zum 'vgl.'-Entfernen.
    \end{itemize}
    \subsection{Wie benutze ich Fußnoten}
    Fußnoten sind noch einfacher\footnote{in der Regel} zu handhaben als die Literatur\footnote{Literatur muss in den Quellen festgehalten werden}.