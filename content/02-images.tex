\subsection{Wie benutze ich Bilder?}
Wenn Sie ein Bild einfügen möchten, nutzen Sie das Konstrukt figure. Beim Konstrukt figure gibt es einige Parameter, um dieses anzupassen.
Sollten Sie das Bild stark anpassen wollen, können Sie im Internet einige hilfreiche Ressourcen finden.
\begin{figure}[H]
    \centering
    \includegraphics[scale=0.5]{beispiel}
    \setcaptioncitation{\citefield{wggwallpaper}{howpublished} \parencite{wggwallpaper}}
    \caption{Ein Beispielbild aus der WGG-Webseite}
    \label{fig:wallpaper}
\end{figure}
Die Platzierung von Abbildungen geschieht bei dem figure-Konstrukt zwischen den Absätzen. Sollte eine Abbildng jedoch am Rand der Arbeit stehen und der Text sich um die Abbildung 'wickeln',
ist das Konstrukt \emph{wrapfigure} zu empfehlen. Ein Beispiel ist hier nebenstehend.
\begin{wrapfigure}{r}{0.4\textwidth}
    \begin{center}
        \includegraphics[scale=0.125]{logo}
        \setcaptioncitation{\citefield{logo}{howpublished} \parencite{logo}}
        \caption{Das Logo des WGG Selb}
    \end{center}
    \label{fig:logo}
\end{wrapfigure}

Ferner kann auch auf ein Bild Bezug genommen werden. Siehe Abbildung \ref{fig:wallpaper}

Dokumentation zum wrapfigure-Package ist ebenfalls online zu finden.
Es kann dabei bestimmt werden, wo die Abbildung platziert werden soll, und welche Breite sie einnehmen darf.
Abgesehen davon sind die Angaben identisch zu jenen, die bei einer normalen \emph{figure} gemacht werden.
