%%%%%%%%%%%%%%%%%%%%%%%%%%%%%%%%%%%%%%%%%%%%%%%%%%%%%%%%%%%%%%%
%                                                             %
%                      (Inoffizielle)                         %
%            LaTeX-Vorlage für eine W-Seminararbeit           %
%              am Walter-Gropius-Gymnasium Selb               %
%                                                             %
%          Erstellt von Justus Müller (Jhrg. 2020)            %
%                                                             %
%%%%%%%%%%%%%%%%%%%%%%%%%%%%%%%%%%%%%%%%%%%%%%%%%%%%%%%%%%%%%%%

% Bevor Sie mit der Ausarbeitung Ihrer Arbeit beginnen,
% füllen Sie bitte die Variablen, die Sie in der Datei unter
%           `./src/parameter.tex`
% finden aus. Nur so ist das Deckblatt Ihrer Arbeit am Ende korrekt.

% Machen Sie sich mit der Bibliographie zu finden unter
%           `./src/quellen.bib`
% vertraut. Um richtig zu zitieren verwenden Sie bitte die Zitierweise
% die Ihnen in diesem Dokument beispielhaft im Kapitel 3.1 gezeigt wird.

% Ergänzen Sie sofern nötig die Datei
%           `./src/packages.tex`
% um weitere Packages, sofern Sie diese benötigen und einbiden wollen.

% Bitte ändern Sie nichts an der Datei
%           `./src/style.tex`
% außer Sie sind sich 100% sicher, was Sie tun.


% Dokumentenklasse
\documentclass{article}

% Includes
% Packages

% Grafiken
\usepackage{graphicx}
\graphicspath{{./img/}}
\usepackage{float}

% Geometrie
\usepackage[a4paper, tmargin=2.5cm, bmargin=2cm, lmargin=2cm, rmargin=2cm]{geometry}

% Soul
\usepackage{soul}

% Fonts
\usepackage{anyfontsize}

% Tikz
\usepackage{tikz}
\usetikzlibrary{positioning}

% Paging
\usepackage{changepage}

% Babel
\ifnum\semLanguage=0
\def\lang{}
\else
    \ifnum\semLanguage=1
    \def\lang{,english}
    \else
        \ifnum\semLanguage=2
        \def\lang{,french}
        \else
        \def\lang{,spanish}
        \fi
    \fi
\fi
\usepackage[ngerman \lang]{babel}

% Hyperref
\usepackage{hyperref}

% Cite style
\ifnum \citeStyle=0
\def\stil{numeric}
\else
\def\stil{authoryear-icomp}
\fi

% BibLaTeX
\usepackage[backend=bibtex, style=\stil]{biblatex}
\addbibresource{./src/sources.bib}
\addbibresource{./src/images.bib}
\usepackage{csquotes}

% Colors
\usepackage{xcolor}

% ToC
\usepackage{tocloft}
\ifnumequal{\numericalTOC}{1}{
    \relax
}{
    \renewcommand{\thesection}{\Alph{section}}
    \renewcommand{\thesubsection}{\Roman{subsection}}
    \renewcommand{\thesubsubsection}{\arabic{subsubsection}}
}
\ifnumequal{\indentTOC}{0}{
    \setlength{\cftsecindent}{0pt}% Remove indent for \section
    \setlength{\cftsubsecindent}{0pt}% Remove indent for \subsection
    \setlength{\cftsubsubsecindent}{0pt}% Remove indent
}{\relax}

% Cite style
\ifnumequal{\citeStyle}{1}{
    \ExecuteBibliographyOptions{autocite=footnote}
}{
    \ifnumequal{\citeStyle}{2}{
        
        \ExecuteBibliographyOptions{autocite=inline}
    }{
        \relax
    }
}

% Captions
\usepackage{caption}
\DeclareCaptionFormat{citation}{%
  \ifx\captioncitation\relax\relax\else
    \captioncitation\par
  \fi
  #1#2#3\par}
\newcommand*\setcaptioncitation[1]{\def\captioncitation{\textsc{\ifnumequal{\semLanguage}{0}{Quelle:}{\ifnumequal{\semLanguage}{1}{Source:}{\ifnumequal{\semLanguage}{2}{Source:}{Fuente:}}}}~#1}}
\let\captioncitation\relax
\captionsetup{format=citation,justification=centering}

% Wrapping figures
\usepackage{wrapfig}

% Maths
\usepackage{amsmath}
%%%%%%%%%%%%%%%%%%%%%%%%%%%%%%%%%%%%%%%%%%%%%%%%%%%%
%     Füllen Sie diese Variablen einmalig aus      %
%    Ihre Seminararbeit wird dann entsprechend     %
%                  generiert.                      %
%%%%%%%%%%%%%%%%%%%%%%%%%%%%%%%%%%%%%%%%%%%%%%%%%%%%

% Variablendefinitionen
\newcommand{\yourthesis}{Mein Thema der Seminararbeit}  % Tragen Sie hier Ihr Thema ein
\newcommand{\guidingtheme}{Das Rahmenthema des W-Seminars}  % Tragen Sie hier das Rahmenthema Ihres W-Seminars ein
\newcommand{\subject}{Beispielfach}  % Tragen Sie hier das Leitfach ein
\newcommand{\myname}{Max Mustermann}  % Tragen Sie hier Ihren Namen ein
\newcommand{\authoridentifier}{Verfasser/in} % Ändern Sie dies zu `Verfasser` oder `Verfasserin` ab
\newcommand{\examyear}{2020}  % Tragen Sie hier Ihren Abiturjahrgang ein
\newcommand{\tutor}{OStRin Maximilia Musterfrau}  % Tragen Sie hier Ihre/n Kursleiter/in (mit Titel) ein
\newcommand{\tutoridentifier}{Kursleiter/in} % Ändern Sie dies zu `Kursleiter` oder `Kursleiterin` ab
\newcommand{\deadline}{5. November 2019}  % Tragen Sie hier den Abgabetermin Ihrer Arbeit ein
\newcommand{\location}{Selb} % Tragen Sie hier Ihren Wohnort ein


% Styleanpassungen

\newcommand{\numericalTOC}{1} % TOC Style
% Hier können Sie sich zwischen einem alpha-numerischen oder einem numerischen Inhaltsverzeichnis (table of contents bzw. TOC) entscheiden
% `1` <-> numerischer TOC
% `0` <-> alpha-numerischer TOC

\newcommand{\citeStyle}{0} % Zitat Style
% Hier können Sie sich zwischen dem Harvard Style mit eckigen Klammern, dem Chicago Style mit Fußnoten,
% oder dem verbose Style (Autor, S. x) entscheiden.
% `0` <-> Harvard Style
% `1` <-> Chicago Style
% `2` <-> verbose Style

\newcommand{\semLanguage}{0}
% Hier können Sie die Sprache Ihrer Seminararbeit wählen. Die Überschriften für das Inhaltsverzeichnis,
% Tabellenverzeichnis und Abbildungsverzeichnis passen sich dementsprechend an.
% `0` <-> Deutsch
% `1` <-> Englisch
% `2` <-> Französisch
% `3` <-> Spanisch

\newcommand{\figureTable}{1}
% Hier können Sie das Abbildungsverzeichnis aktivieren oder deaktivieren
% `0` <-> Aus
% `1` <-> An

\newcommand{\tableTable}{1}
% Hier können Sie das Tabellenverzeichnis aktivieren oder deaktivieren
% `0` <-> Aus
% `1` <-> An

% Styling


% DO NOT TOUCH, IF YOU DON'T KNOW WHAT YOU'RE DOING!


\newcommand{\setupSeminararbeit}{
    % Titelseite
    \begin{titlepage}
        \newgeometry{bottom=0.1cm}

        % Kopfzeile
        \thispagestyle{empty}
        \setlength{\voffset}{-2.5cm}
        \hspace{-1.3cm}
        \begin{minipage}{0.65\textwidth}
            \includegraphics[width=50mm]{logo}
        \end{minipage}
        \begin{minipage}{0.42\textwidth}
            \rightline{Abiturjahrgang \examyear}
        \end{minipage}
        
        
        % Titel
        \begin{center}
            \vspace{2.5cm}
            \fontsize{14}{0}\selectfont
            \so{\textbf{SEMINARARBEIT}}\\
            \vspace{1.5cm}
            \fontsize{11}{16}\selectfont
            Rahmenthema des Wissenschaftspropädeutischen Seminars:\\
            \emph{\guidingtheme{}}\\
            Leitfach: \subject{}\\
            \vspace{2cm}
            \fontsize{12}{16}\selectfont
            Thema der Arbeit:\\
            \emph{\yourthesis{}}
        \end{center}
        \vspace{3.5cm}
        \fontsize{11}{18}\selectfont
        \begin{tabular}{l @{\hspace{1cm}} l}
            Verfasser/in: & \myname{} \\
            Kursleiter/in: & \tutor{} \\
            Abgabetermin: & \deadline{} \\
        \end{tabular} \\[0.8cm]
        
        % Tabelle
        \begin{adjustwidth}{-0.5cm}{0cm}
            \begin{tikzpicture}[node distance = 0pt, every node/.style = {draw}]
                \node (h1) [minimum width = 4cm, minimum height = 0.6cm, font = \bf] {Bewertung};
                \node (h2) [right = of h1, minimum height = 0.6cm, minimum width = 2cm] {Note};
                \node (h3) [right = of h2, minimum height = 0.6cm, minimum width = 5cm] {Notenstufe in Worten};
                \node (h4) [right = of h3, minimum height = 0.6cm, minimum width = 1.8cm] {Punkte};
                \node (h5) [right = of h4, minimum height = 0.6cm, minimum width = 1cm] {};
                \node (h6) [right = of h5, minimum height = 0.6cm, minimum width = 1.8cm] {Punkte};
    
                \draw[-] (h5.north west) -- (h5.south east);
    
                \node (s1) [below = of h1, minimum height = 1.2cm, minimum width = 4cm] {schriftliche Arbeit};
                \node (s2) [below = of h2, minimum height = 1.2cm, minimum width = 2cm] {};
                \node (s3) [below = of h3, minimum height = 1.2cm, minimum width = 5cm] {};
                \node (s4) [below = of h4, minimum height = 1.2cm, minimum width = 1.8cm] {};
                \node (s5) [below = of h5, minimum height = 1.2cm, minimum width = 1cm] {x3};
                \node (s6) [below = of h6, minimum height = 1.2cm, minimum width = 1.8cm] {};

                \node (a1) [below = of s1, minimum height = 1.2cm, minimum width = 4cm] {Abschlusspräsentation};
                \node (s2) [below = of s2, minimum height = 1.2cm, minimum width = 2cm] {};
                \node (s3) [below = of s3, minimum height = 1.2cm, minimum width = 5cm] {};
                \node (s4) [below = of s4, minimum height = 1.2cm, minimum width = 1.8cm] {};
                \node (s5) [below = of s5, minimum height = 1.2cm, minimum width = 1cm] {x1};
                \node (s6) [below = of s6, minimum height = 1.2cm, minimum width = 1.8cm] {};

                \node (sigma6) [below = of s6, minimum height = 1.2cm, minimum width = 1.8cm] {};
                \node (sigma5) [left = of sigma6, draw = none] {Summe:};

                \node (total6) [below = of sigma6, minimum height = 1.2cm, minimum width = 1.8cm] {};
                \node (total5) [left = of total6, draw = none] {Gesamtleistung nach §61 (7) GSO = Summe : \textbf{2} (gerundet):};
            \end{tikzpicture}
        \end{adjustwidth}
        
        \vspace{1cm}

        % Unterschift
        \par\noindent\rule{\textwidth}{0.3pt}
        Datum und Unterschrift der Kursleiterin bzw. des Kursleiters
    \end{titlepage}

    % Inhaltsverzeichnis
    \restoregeometry{}
    \fontsize{11}{24}\selectfont
    \setcounter{page}{2}
    \tableofcontents
    \newpage

    % Schriftartanpassung
    \fontsize{11}{19}\selectfont

}


\newcommand{\insertOfficialDocs}{
     % Bibliographie
     \subsection{Quellenangaben}
     \printbibliography[notkeyword={picture},heading=subbibliography,title={Literaturverzeichnis}]
     \printbibliography[keyword={picture}, heading=subbibliography, title={Abbildungsverzeichnis}]
     \newpage
 
 
     % Eigenständigkeitserklärung
     \subsection{Eigenständigkeitserklärung}
     Hiermit erkläre ich, dass ich die vorliegende Arbeit selbständig und ohne fremde Hilfe verfasst und keine anderen als die angegebenen Hilfsmittel verwendet habe. \\
     Insbesondere versichere ich, dass ich alle wörtlichen und alle sinngemäßen Übernahmen aus anderen Werken als solche kenntlich gemacht habe. Auch alle genutzten Internetseiten wurden kenntlich gemacht und sind im Verzeichnis der Internetseiten bzw. im Quellenverzeichnis aufgeführt. \\
     Ich bin darüber belehrt worden, sofern sich – auch zu einem späteren Zeitpunkt – herausstellen sollte, dass die Seminararbeit oder Teile davon nicht selbständig verfasst wurden, die Hinweise zu Zitaten fehlen oder Teile unverändert aus dem Internet übernommen wurden, die Arbeit auch nachträglich mit der Note ungenügend gewertet wird. \\[1cm]
 
     \begin{tabular}{@{}p{10cm}p{5cm}@{}}
         \location, den \today & \hrulefill \\
         & \hspace{1.5cm}Unterschrift \\
     \end{tabular}
     \newpage
}

% Dokument
\begin{document}

    \setupSeminararbeit{}
    


    % Beispielkapitel
    \section{Einleitung}
    \subsection{Prolog}
    Lorem ipsum dolor sit amet, consetetur sadipscing elitr, sed diam nonumy eirmod tempor invidunt ut labore et dolore magna aliquyam erat, sed diam voluptua.
    At vero eos et accusam et justo duo dolores et ea rebum.
    Stet clita kasd gubergren, no sea takimata sanctus est Lorem ipsum dolor sit amet.
    Lorem ipsum dolor sit amet, consetetur sadipscing elitr, sed diam nonumy eirmod tempor invidunt ut labore et dolore magna aliquyam erat, sed diam voluptua.ä
    At vero eos et accusam et justo duo dolores et ea rebum. Stet clita kasd gubergren, no sea takimata sanctus est Lorem ipsum dolor sit amet.

    \subsection{Bezug zu anderen Veröffentlichungen}
    Hier gibt es bald viel zu sehen!
    \newpage




    % Beispielkapitel mit Bildern
    \section{Bilder}
    \subsection{Wie benutze ich Bilder?}
    Wenn Sie ein Bild einfügen möchten, nutzen Sie das Konstrukt figure. Beim Konstrukt figure gibt es einige Parameter um dieses anzupassen.
    Sollten Sie das Bild stark anpassen wollen, können Sie im Internet einige hilfreiche Ressourcen finden.
    Wenn Sie Text um eine Abbildung herum platzieren möchten, ist möglicherweise ein Bild in das Konstrukt minipage oder das Paket float eine gute Idee.
    Ich wünsche Ihnen viel Erfolg beim formatieren Ihrer Bilder. (Schlimmer als Word wird es nicht!)\\
    \begin{figure}[H]
        \includegraphics[scale=0.5]{beispiel}
        \caption{Ein Beispielbild aus der WGG-Webseite \parencite{wggwallpaper}}
        \label{wallpaper}
    \end{figure}
    Lorem ipsum dolor sit amet, consetetur sadipscing elitr, sed diam nonumy eirmod tempor invidunt ut labore et dolore magna aliquyam erat, sed diam voluptua.
    At vero eos et accusam et justo duo dolores et ea rebum.
    Stet clita kasd gubergren, no sea takimata sanctus est Lorem ipsum dolor sit amet.
    Lorem ipsum dolor sit amet, consetetur sadipscing elitr, sed diam nonumy eirmod tempor invidunt ut labore et dolore magna aliquyam erat, sed diam voluptua.ä
    At vero eos et accusam et justo duo dolores et ea rebum. Stet clita kasd gubergren, no sea takimata sanctus est Lorem ipsum dolor sit amet.
    \newpage



    % Beispielkapitel mit Fußnoten
    \section{Fußnoten und Quellen}
    \subsection{Wie zitiere ich?}
    Das Zitieren einer Quelle wird beispielsweise so gemacht: Im Video von SimpleClub \parencite{simpleclub} wird der schiefe Wurf erklärt.
    Auch in einem Buch wird das gleiche Phänomen beschrieben \parencite[vgl.][]{ruprecht} . \\
    Es funktioniert in \LaTeX{} so:
    \begin{itemize}
        \item Schreiben Sie Ihre Quelle in die \texttt{quellen.bib}-Datei
        \item Zitieren Sie die Quelle im Text mit \begin{verbatim}
            \parencite{quelle}
        \end{verbatim}
        Das ist gar nicht so schwer :)
    \end{itemize}
    \subsection{Wie benutze ich Fußnoten}
    Fußnoten sind noch einfacher\footnote[1]{also in der Regel} zu handhaben als die Literatur\footnote[2]{Literatur muss in den Quellen festgehalten werden}.
    \newpage



    % Beispielkapitel mit Formeln
    \section{Mathematik}
    \subsection{Formeln}
    Wir besprechen kurz wie Formeln eingefügt werden können.
    Dies ist eine Formel, die im Textfluss auftaucht $a^{2} + b^{2} = c^{2}$.
    Diese Formel ist isoliert:
    \begin{center}
        $x_1,_2 = \frac{-b \pm \sqrt{b^2 - 4ac}}{2a}$
    \end{center}



    % Beispielsektion Diagramme
    \subsection{Diagramme}
    Hier ein kurzes Beispiel eines Diagramms, dass mit tikz erstellt wurde.\\[1cm]
    \begin{tikzpicture} [set/.style = {draw,
        circle,
        minimum size = 6cm,
        fill=blue,
        opacity = 0.4,
        text opacity = 1}]
     
    \node (A) [set] {wichtige Inhalte};
    \node (B) at (60:4cm) [set] {wissenschaftlich};
    \node (C) at (0:4cm) [set] {angenehm zu lesen};
     
    \node at (barycentric cs:A=1,B=1) [left] {Paper};
    \node at (barycentric cs:A=1,C=1) [below] {Zeitung};
    \node at (barycentric cs:B=1,C=1) [right] {xkcd.com};
    \node at (barycentric cs:A=1,B=1,C=1) [] {Seminararbeit};
     
    \end{tikzpicture}
    Sie können natürlich auch Diagramme extern erzeugen und als Bild einfügen. Außerdem steht Ihenen bei \href{https://www.mathcha.io/}{mathcha}\footnote[1]{ \href{https://www.mathcha.io/}{https://www.mathcha.io/}} ein Editor für Diagramme und Text zur verfügung.\\
    Wenn Sie Anfangsschwierigkeiten mit \LaTeX{} haben, kann ich Ihnen die Zurhilfenahme dieses Services nur ans Herz legen, da die Dokumente als TeX-Datei exportiert werden können.
    \newpage




    % Schlusswort
    \section{Schlusswort}
    \subsection{Fazit}
    Meine Seminararbeit war erfolgreich weil \dots
    \newpage


    % Bibliographie und Eigenständigkeitserklärung
    \insertOfficialDocs{}


    % Anhang
    \section{Anhang}
    Hier ist Platz für Ihren Anhang.



\end{document}