%%%%%%%%%%%%%%%%%%%%%%%%%%%%%%%%%%%%%%%%%%%%%%%%%%%%
%     Füllen Sie diese Variablen einmalig aus      %
%    Ihre Seminararbeit wird dann entsprechend     %
%                  generiert.                      %
%%%%%%%%%%%%%%%%%%%%%%%%%%%%%%%%%%%%%%%%%%%%%%%%%%%%

% Variablendefinitionen
\newcommand{\yourthesis}{Mein Thema der Seminararbeit}  % Tragen Sie hier Ihr Thema ein
\newcommand{\guidingtheme}{Das Rahmenthema des W-Seminars}  % Tragen Sie hier das Rahmenthema Ihres W-Seminars ein
\newcommand{\subject}{Beispielfach}  % Tragen Sie hier das Leitfach ein
\newcommand{\myname}{Max Mustermann}  % Tragen Sie hier Ihren Namen ein
\newcommand{\authoridentifier}{Verfasser/in} % Ändern Sie dies zu `Verfasser` oder `Verfasserin` ab
\newcommand{\examyear}{2020}  % Tragen Sie hier Ihren Abiturjahrgang ein
\newcommand{\tutor}{OStRin Maximilia Musterfrau}  % Tragen Sie hier Ihre/n Kursleiter/in (mit Titel) ein
\newcommand{\tutoridentifier}{Kursleiter/in} % Ändern Sie dies zu `Kursleiter` oder `Kursleiterin` ab
\newcommand{\deadline}{5. November 2019}  % Tragen Sie hier den Abgabetermin Ihrer Arbeit ein
\newcommand{\location}{Selb} % Tragen Sie hier Ihren Wohnort ein


% Styleanpassungen

\newcommand{\numericalTOC}{1} % TOC Style
% Hier können Sie sich zwischen einem alpha-numerischen oder einem numerischen Inhaltsverzeichnis (table of contents bzw. TOC) entscheiden
% `1` <-> numerischer TOC
% `0` <-> alpha-numerischer TOC

\newcommand{\citeStyle}{0} % Zitat Style
% Hier können Sie sich zwischen dem Harvard Style mit eckigen Klammern, dem Chicago Style mit Fußnoten,
% oder dem verbose Style (Autor, S. x) entscheiden.
% `0` <-> Harvard Style
% `1` <-> Chicago Style
% `2` <-> verbose Style

\newcommand{\semLanguage}{0}
% Hier können Sie die Sprache Ihrer Seminararbeit wählen. Die Überschriften für das Inhaltsverzeichnis,
% Tabellenverzeichnis und Abbildungsverzeichnis passen sich dementsprechend an.
% `0` <-> Deutsch
% `1` <-> Englisch
% `2` <-> Französisch
% `3` <-> Spanisch

\newcommand{\figureTable}{1}
% Hier können Sie das Abbildungsverzeichnis aktivieren oder deaktivieren
% `0` <-> Aus
% `1` <-> An

\newcommand{\tableTable}{1}
% Hier können Sie das Tabellenverzeichnis aktivieren oder deaktivieren
% `0` <-> Aus
% `1` <-> An
